\section{Full adder as network}\label{app:Full_adder}
To create a full adder from basic neurons, the corresponding logic gates need to be defined. The equivalent neuron for an ''and''-gate was defined in \autoref{sec:basics_neuron_network}. There are two more basic neurons which need to be defined. The neuron corresponding to the ''or''-gate, which is given by the same activation function \eqref{def:step_activation}, weights $\vec{w}={(w_1, w_2)}^T=(1,1)$ and bias $b=-0.5$, and the equivalent neuron for the ''not''-gate, which is given by the activation function \eqref{def:step_activation}, weight $w=-1$ and bias $b=0.5$. These definitions are summarized in \autoref{tab:neuron_logic_gates}.
\begin{table}[H]
\begin{center}
\begin{tabular}{c|c|c}
''and''-neuron & ''or''-neuron & ''not''-neuron\\
\hline
\input{tikzgraphics/tikz_and_neuron} & \begin{tikzpicture}[
neuron/.style={circle, draw=black, very thick, minimum size=1.0cm},
dot/.style={circle, draw=black, fill=black, minimum size=0.1cm, inner sep=0pt},
VLineVertex/.style={circle, draw=black, minimum size=0cm, inner sep=0pt},
]

\node[neuron] (neuron) {\textbf{or}};
\node (place) [left=1cm of neuron] {};
\node (x1) [above=0.25cm of place] {$x_1$};
\node (x3) [below=0.25cm of place] {$x_3$};
\node (out) [right=0.75cm of neuron] {$x_1\lor x_2$};

\draw (x1.east) -- (neuron.west);
\draw (x3.east) -- (neuron.west);
\draw [->] (neuron.east) -- (out.west);

\end{tikzpicture} & \begin{tikzpicture}[
neuron/.style={circle, draw=black, very thick, minimum size=1.0cm},
dot/.style={circle, draw=black, fill=black, minimum size=0.1cm, inner sep=0pt},
VLineVertex/.style={circle, draw=black, minimum size=0cm, inner sep=0pt},
]

\node[neuron] (neuron) {\textbf{not}};
\node (x1) [left=0.65cm of neuron] {$x_1$};
\node (out) [right=0.65cm of neuron] {$\lnot x_1$};

\draw (x1.east) -- (neuron.west);
\draw [->] (neuron.east) -- (out.west);

\end{tikzpicture}\\
\hline
\begin{tabular}{c c}
$\vec{w}=(1,1)$ & $b=-1.5$
\end{tabular} &
\begin{tabular}{c c}
$\vec{w}=(1,1)$ & $b=-0.5$
\end{tabular} &
\begin{tabular}{c c}
$w=-1$ & $b=0.5$
\end{tabular}\\
\hline
\begin{tabular}{c c|c}
$x_1$ & $x_2$ & $a(x_1+x_2-1.5)$\\
\hline
$0$ & $0$ & $0$\\
$0$ & $1$ & $0$\\
$1$ & $0$ & $0$\\
$1$ & $1$ & $1$\\
\end{tabular} &
\begin{tabular}{c c|c}
$x_1$ & $x_2$ & $a(x_1+x_2-0.5)$\\
\hline
$0$ & $0$ & $0$\\
$0$ & $1$ & $1$\\
$1$ & $0$ & $1$\\
$1$ & $1$ & $1$\\
\end{tabular} &
\begin{tabular}{c|c}
$x_1$ & $a(-x_1+0.5)$\\
\hline
$0$ & $1$\\
$1$ & $0$\\
\end{tabular}
\end{tabular}
\end{center}
\caption{A summary and depiction of the main logic gates written as neurons. All of them share the same activation function \eqref{def:step_activation}.}\label{tab:neuron_logic_gates}
\end{table}
\medskip
\noindent Using the basic logic gates a more complex structure - the ''XOR''-gate - can be built. A ''XOR''-gate is defined by its truth table (see \autoref{tab:xor}).
\begin{table}[H]
\centering
\begin{tabular}{c c|c}
$x_1$ & $x_2$ & $x_1\veebar x_2$\\
\hline
$0$ & $0$ & $0$\\
$0$ & $1$ & $1$\\
$1$ & $0$ & $1$\\
$1$ & $1$ & $0$
\end{tabular}
\caption{Truth table for the ''XOR''-gate.}\label{tab:xor}
\end{table}
It can be constructed from the three basic logic operations ''and'', ''or'' and ''not''
\begin{equation}
x_1\veebar x_2 = \lnot\lr{\lr{x_1\land x_2}\lor\lnot\lr{x_1\lor x_2}}.
\end{equation}
Therefore the basic neurons from \autoref{tab:neuron_logic_gates} can be combined to create a ''XOR''-network (see \autoref{fig:xor_net}).\\
To simplify readability from here on out a neuron called ''XOR'' will be used. It is defined by the network of \autoref{fig:xor_net} and has to be replaced by it, whenever it is used.\\
With this ''XOR''-neuron a network, that behaves like a full-adder, can be defined. A full-adder is a binary adder with carry in and carry out, as seen in \autoref{fig:full_adder}.
\begin{figure}[H]
\centering
\begin{tikzpicture}[
neuron/.style={circle, draw=black, very thick, minimum size=1.0cm},
dot/.style={circle, draw=black, fill=black, minimum size=0.1cm, inner sep=0pt},
VLineVertex/.style={circle, draw=black, minimum size=0cm, inner sep=0pt},
]

\node[neuron] (out_not) {\textbf{not}};
\node[neuron] (final_or) [left=1cm of out_not] {\textbf{or}};
\node (between) [left=1cm of final_or] {};
\node[neuron] (bot_not) [below=0.25cm of between] {\textbf{not}};
\node[neuron] (bot_or) [left=1cm of bot_not] {\textbf{or}};
\node[neuron] (x2) [left=1cm of bot_or] {$x_2$};
\node[neuron] (x1) [above=0.5cm of x2] {$x_1$};
\node[neuron] (top_and) [right=2cm of x1] {\textbf{and}};
\node (out) [right=0.5cm of out_not] {$x_1\veebar x_2$};

%Input
\draw (x1.east) -- (top_and.west);
\draw (x2.east) -- (top_and.west);
\draw (x1.east) -- (bot_or.west);
\draw (x2.east) -- (bot_or.west);

%First and second layer connections
\draw (top_and.east) -- (final_or.west);
\draw (bot_or.east) -- (bot_not.west);
\draw (bot_not.east) -- (final_or.west);

%Final connection
\draw (final_or.east) -- (out_not.west);

%Output
\draw [->] (out_not.east) -- (out.west);
\end{tikzpicture}
\caption{The definition of a network that is equivalent to an ''XOR''-gate.}\label{fig:xor_net}
\end{figure}
\begin{figure}[H]
\centering
\begin{tikzpicture}[
neuron/.style={circle, draw=black, very thick, minimum size=1.4cm},
dot/.style={circle, draw=black, fill=black, minimum size=0.1cm, inner sep=0pt},
VLineVertex/.style={circle, draw=black, minimum size=0cm, inner sep=0pt},
]

\node[neuron] (x1) {$x_1$};
\node[neuron] (x2) [below=0.5cm of x1] {$x_2$};
\node[neuron] (x3) [below=0.5cm of x2] {$x_3$};
\node (center_xor1) [below=0.25cm of x1] {};
\node[neuron] (xor1) [right=1.5cm of center_xor1] {\textbf{XOR}};
\node[neuron] (xor2) [right=4.1cm of x2] {\textbf{XOR}};
\node[neuron] (and1) [below=0.5cm of xor2] {\textbf{and}};
\node[neuron] (and2) [below=0.5cm of and1] {\textbf{and}};
\node (center_or1) [below=0.25cm of and1] {};
\node[neuron] (or1) [right=1cm of center_or1] {\textbf{or}};
\node (out_sum) [right=2.5cm of xor2] {Sum};
\node (out_car) [right=0.5cm of or1] {$\text{Carry}_\text{out}$};

%Inputs
\node (x1_in) [left=0.5cm of x1] {$x_1$};
\node (x2_in) [left=0.5cm of x2] {$x_2$};
\node (car_in) [left=0.5cm of x3] {$\text{Carry}_\text{in}$};

%Draw lines
%Input arrows
\draw[->] (x1_in.east) -- (x1.west);
\draw[->] (x2_in.east) -- (x2.west);
\draw[->] (car_in.east) -- (x3.west);

%Input connections
\draw (x1.east) -- (xor1.west);
\draw (x1.east) -- (and2.west);
\draw (x2.east) -- (xor1.west);
\draw (x2.east) -- (and2.west);
\draw (x3.east) -- (xor2.west);
\draw (x3.east) -- (and1.west);

%xor1 connections
\draw (xor1.east) -- (xor2.west);
\draw (xor1.east) -- (and1.west);

%Third layer connections
\draw[->] (xor2.east) -- (out_sum.west);
\draw (and1.east) -- (or1.west);
\draw (and2.east) -- (or1.west);

%Or out
\draw[->] (or1.east) -- (out_car.west);
\end{tikzpicture}
\caption{A network replicating the behavior of a binary full adder.}\label{fig:full_adder}
\end{figure}