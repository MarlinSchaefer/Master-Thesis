\section{Tested networks}
\textcolor{blue}{The title needs to be improved!\\Explain what we are trying to do once more and how our net achieves that.}
This section summarizes the results of our research, trying to find a deep learning algorithm that succeeds in detecting and classifying \gls{bns}-signals in noisy data. The goal was to not only classify the signals into the two categories ''pure noise'' and ''signal'' but also give an estimate of the signals \gls{snr}.\\
All software written for this work used the PyCBC software package \cite{pycbc} for any code involving gravitational wave data and the deep learning library Keras \cite{keras} to rapidly develop and test different network architectures. We used Tensorflow \cite{tensorflow} as the backend for Keras. \textcolor{red}{Cite specific version numbers.}
\subsection{Motivation}
\textcolor{blue}{Motivate why the problem we are looking at is interesting and why it might be beneficial to train for SNR instead of simply classifying. Why is BNS more difficult? (Care, I think I wrote a little bit about this in the Data Generating Process section) Maybe drop this if the motivation is clear from the introduction and related works section.}
\subsection{The Data Generating Process}
\textcolor{red}{Things missing: Mention training/validation split, mention that we shift the template around in the data (important for why we slide the generator in steps of 0.25s). ATTENTION!!! THE GENERATOR USED FOR THE TESTING SET CURRENTLY USES THE ANALYTIC PSD INSTEAD OF THE WHITEN FUNCTION! THIS IS DISPLAYED DIFFERENTLY IN THE TEXT! CORRECT EITHER THE GENERATOR OR THE TEXT! Take great care to carefully explain multirate filtering.}\\
\textcolor{blue}{Explain how the data is generated (especially that it only contains gaussian noise). Explain how training/validation set are different from the testing set used in this work. Explain how we carefully chose the parameters, especially how we got the SNR-values. Also explain the mix-and-match approach, that noise can be pure or the same noise with different signals. Also the same signal can (but this isn't necessary) have multiple noise instances. The pure noise SNR was set to 4, as that is the average value for pure noise matched filtering returns. (One might want to alter this) Finally we were careful, that no noise instance or signal from the training set is used in the validation set.}\\
To train a \gls{nn} to detect and classify \gls{bns} signals, we need a large set of mock data. \gls{bns} signals are more difficult to train for than \gls{bbh} signals, as their signals are a lot weaker and last for a longer duration\footnote{\gls{bbh} signals spend about \SI{1}{\s} within the sensitive frequency range of our detectors, whereas \gls{bns} signals can be visible in the whitened detector data for multiple tens of seconds \cite{gw170817} and when generated even last multiple hundred seconds.}. Training a \gls{nn} on such large data, that is sampled sufficiently to resolve the highest frequencies occurring during the merger \textcolor{red}{(Need to introduce Inspiral/Merger/Ringdown in 2.1)}, is not feasible. As we want to loose as little \gls{snr} as possible, we also do not want to crop the data to contain only the merger and very little of the inspiral.\\
Although \gls{bns} signals spend a long time in sensitive region of the detectors, their frequency evolution for large parts of the signal is rather slow, as a lot more of the inspiral is visible. The inspiral having larger frequencies is due to the lower masses involved. (Compare \eqref{def:frequency_evolution_differential} and \eqref{def:frequency evolution}) For the inspiral-part of the signal, frequencies are around \SIrange{30}{100}{\hertz}. To resolve these frequencies, a sample rate of \SIrange{60}{200}{\hertz} is necessary. The higher sample rates are only required for the final few seconds. For these reasons we introduce a new multi-rate approach. The network does not receive the data at a fixed sample-rate, but rather multiple inputs, each sampling parts of the signal at different rates.\\
We choose the window to encompass \SI{64}{\s} of data, where a signal is aligned such, that the merger is roughly \SI{0.5}{\s} from the end of the data. Afterwards we chop the data into parts of duration \SI[parse-numbers=false]{2^i}{\s} and re-sample each of these parts to have a sample rate of \SI[parse-numbers=false]{2^{12-i}}{\hertz}, with $i=0,\dotsc ,6$. This way each sample rate contains exactly $4096$ samples.\\
Each sample rate however overlaps with the higher sample rates, e.g. the \SI{2}{\s} interval sampled at \SI{2048}{\hertz} also includes the \SI{1}{\s} interval sampled at \SI{4096}{\hertz}. For this reason and to reduce the number of input samples even further, we only use the first $2048$ samples for each rate, except the highest sample rate. To keep things simple however, the highest sample rate is split into two parts, each containing $2048$ samples. Therefore each \SI{64}{\s} input interval is split and re-sampled to yield 8 inputs, each containing $2048$ samples. \textcolor{red}{(Include graphic visualizing multi-rate-filtering)}
\medskip\\
To obtain a large set of training data, fake data is generated by utilizing the PyCBC software package \cite{pycbc}. \textcolor{red}{Check which version I'm using and cite that one.} The final training set contained $56250$ different signals and $161250$ different noise realizations. All waveforms were generated using the approximant ''TaylorF2'', as implemented by PyCBC. \textcolor{red}{(Should this be Lal?)} Out of the 17 parameters that could have been varied, we chose to fix the spins to be $0$ and neglected tidal effects for simplicity. Furthermore the coalescence time $t_\text{coal}$ is set to be $0$. The remaining $8$ parameters were chosen to represent a realistic distribution, in order to estimate the potential of our approach in a real search. As such, both component masses $m_1$ and $m_2$ are uniformly distributed in the range \SIrange{1.2}{1.6}{M_\odot} each. Specifically we do not explicitly require $m_1\geq m_2$ when generating the waveform. The coalescence phase $\Phi_0$  and the polarization angle $\psi$ are uniformly distributed on the interval $\left[0, 2\pi\right]$ and the inclination $\iota$ is distributed like $\arccos\lr{\text{uniform}\lr{-1, 1}}$. Finally the sky-position is isotropic, i.e. $\theta$ is distributed like $\arccos\lr{\text{uniform}\lr{-1, 1}}$ and $\varphi$ uniform in $\left[-\pi, \pi\right]$.\\
The luminsoity distance $r$ is not chosen directly, but indirectly by fixing the \gls{snr} to some value. This is valid, as the \gls{snr} scales inversely with the distance. \textcolor{red}{(Is this true, or is for instance $SNR\approx 1/r^2$?))} In this work, the \gls{snr} is uniformly distributed on the interval $\left[8,15\right]$. One has to avoid one major pitfall when fixing or calculating the \gls{snr} and comparing it to other results. If one compares the \gls{snr} of two signals with the same parameters, the value will heavily depend on the length of the segment used. According to \textcolor{red}{Reference to matched filtering section}, cutting off the waveform early might result in a lower or at least inaccurate value of the \gls{snr}. For this reason, to make meaningful statements about the \gls{snr} of a signal, we need to specify how we calculated it. For this work, we generate the waveforms with a lower frequency cutoff of \SI{20}{\hertz}, which results in waveforms, with a duration of about \SI{500}{\s}. After these waveforms are generated, we calculate the projection onto the detectors and crop the signals in such a way, that they span \SI{96}{\s} and the merger lies within the last second. Only after the waveforms are cropped, we calculate the \gls{snr} with no noise present (while assuming the \gls{psd} of the detector) using the waveform itself. Since we are using multiple detectors, this value $\rho_i$ is calculated for each detector. The total \gls{snr} in the absence of noise is given by
\begin{equation}
\rho_\text{total} = \sqrt{\sum_i\rho_i^2}.
\end{equation}
Each waveform is than rescaled by multiplying with the factor $\text{\gls{snr}} / \rho_\text{total}$. As a last step, before re-sampling the data as described above, the waveforms are whitened. For the training and validation set this whitening procedure is done using the analytic \gls{psd} \verb|aLIGOZeroDetHighPower| provided by PyCBC, instead of using an estimate of the \gls{psd} based on the data itself. The testing set however uses the estimate of the \gls{psd}, and not the analytic form. The reason to treat these two sets differently from the testing set, is the way, the samples are fed to the network. For the training/validation set, we store the noise samples and pure signals separately and only add them as a first step of the network. For this reason, an estimate of the \gls{psd} would be meaningless at least for the pure signals. For the testing set however we want to mimic, how a pipeline of a detector might work. For this scenario using an analytic model would not be accurate. Therefore in this step the \gls{psd} is only estimated.\\
The reason for storing noise and signals separately are resource constraints. To cover the entire parameter-space densely enough and avoid overfitting, a large number of samples are necessary. Initially we generated and stored the sum of signal and noise, instead of storing each category separately. This has multiple disadvantages, but also one key advantage; we can use the pure signal as a filter for the optimal filter \textcolor{red}{(Insert equation number here)} and give a best-case recovery \gls{snr}, that a search could return. In that sense, we could monitor the performance and compare it to matched filtering directly during training. The disadvantages however at some point outweighed this advantage. The core disadvantage was the restricted number of samples. A file containing $500,000$ samples has a file-size on the order of \SI{200}{\giga\byte}. To train the networks on the data, we completely load it into system memory and need some overhead for formatting. To reduce these costs, we decided to split the signal- and noise-samples and only at runtime choose one instance of each category, which are than added together on the first layer of the network. The second advantage of this approach is less obvious. It enables us to easily feed the network the same signal submerged in multiple different noise realizations, which resulted in performance improvements for similar tasks (Christoph Dreißigacker, personal communication, June 2019).\\
The split between training and validation set is treated with great care, assuring, that not a single noise or signal sample from the training set is used during validation. Therefore the reported loss and accuracy values are representative of a real search. Though they are not the final statistic, we report, they are tightly linked to those and give clues about the network and its efficiency.\\
The data used for training and validating the final network contained $75,000$ different \gls{gw}-signals and $215,000$ noise realizations. We than generate a set number of unique index pairs $(s_i, n_i)$, where $s_i$ corresponds to a signal and $n_i$ to a noise sample. For the training set these indices may be selected from $s_i\in\left[0, 3/4s_t\right)$ and $n_i\in\left[0, 3/4n_t\right)$, where $s_t$ and $n_t$ are the total number of signals and noise samples respectively. If $3/4s_t\notin\mathcal{N}$ or $3/4n_t\notin\mathcal{N}$, the upper index is rounded to the nearest natural number. The total number of pairs generated is equal to the number of usable noise samples $3/4 n_t$. These index pairs represent all samples of the training/validation set that contain a \gls{gw}. In order to also supply pure noise samples to the learning algorithm, all noise realizations are also used during training. This is achieved by appending all index pairs $\lr{-1, 0}, \lr{-1, 1}, \dotsc, \lr{-1, 3/4n_t}$ to the list of index pairs generated before. Afterwards this list is shuffled. The \gls{nn} finally is fed with these $2\cdot 3/4 n_t$ samples through a function\footnote{Keras calls this function a generator.} that interprets the indices and reshapes the data.\\
The shape of the data depends on the network in use. Our final network expects a list of $16$ arrays, where each array is of shape $\lr{\text{mini-batch size, 2048, 2}}$. The last axis is the number of detectors used, whereas the second axis is the time series strain data. We generate the data for the two \gls{ligo} detectors Hanford and Livingston. \textcolor{red}{(Cite papers about these detectors. Is LIGO introduced as abbreviation?)}\\
At this point we stress again, that we use only simulated data. As such all noise that is used follows an analytic \gls{psd} \textcolor{red}{(Maybe cite the paper that gives the PSD, otherwise mention again how the PSD is called in PyCBC?)} and is gaussian. Hence no rigorous statements on the real world performance can be made.\medskip\\
To still report meaningful results, we took care of separating the data generating process for training/validation and testing set. These two sets use two different pipelines to generate the data. The testing data is supplied as continuous time series data, cut into large chunks. This chunk is than handed to a function\footnote{Again, this is a generator in Keras terms.} that takes \SI{64}{\s}

\subsection{Evolution of the Architecture}
\textcolor{blue}{Talk about the different things we tried, what didn't work and how we improved on them. (Improvement from convolution to inception, improvement from inception to collect-inception, improvement from inception to tcn-inception, improvement of tcn-collect-inception over collect-inception.)}
This section gives an overview of the process that led to the final architecture. It will chronologically highlight the pivotal points along the way and showcase some ideas that did not work out. \textcolor{red}{A detailed timeline can be found in \cite{network_wiki}. (Should the wiki be turned into a PDF and migrated to github?)}\\
As a starting point we tried to use an easy case, where the \gls{snr} was uniformly distributed between 10 and 50, with all other parameters fixed. The neutron stars were modeled with $m_1=m_2=$ \SI{1.4}{M_\odot}. The data was stored and loaded as the sum of signal and noise and contained data for the two detectors Hanford and Livingston. In general, the data contained samples of signals and pure noise realizations. Until stated otherwise, all of the following networks use this data.\\
To rate the performance of a network we didn't use the sensitivity of the network yet. Instead we used the variance and mean squared error of the recovered \gls{snr}-values to estimate performance. The hope was that these simple statistics correlate strongly with the actual sensitivity.\medskip\\
The first architecture we used was very close in nature to that of \cite{huerta_parameter_estimation}, halving, however, the number of filters in each convolution layer and using batch normalization in between the convolution and activation. Therefore, we used a network of 3 stacked convolution layers, each followed by a batch normalization, ReLU activation \textcolor{red}{(Never introduced the ReLU activation)} and maximum pooling layer. The number of filters was doubled after each convolution layer. Since the input data to our network is filtered at multiple rates, this network has 14 input channels instead of 2. (7 input channels per detector, where each of the 7 channels corresponds to a single sample rate)\\
Though trained without pure noise samples and with only the \gls{snr} as a training goal, we soon changed to use the data as mentioned in the beginning of this section and training also for a second output. This second output gives a number between 0 and 1, where 1 corresponds to the network classifying the data as signal and 0 for classifying it as pure noise. All of the following analysis is based purly on (11.02.2019) in \cite{network_wiki} and its limited information. Therefore, a lot of the statements below are only qualitative rather than quantitative.\\
The network was able to recover the \gls{snr} of signals rather well but has a large spread for the recovered \gls{snr} values of pure noise samples. The sensitivity can be eyeballed to be reach 100\% only after around \gls{snr} $\sim 25$ and dropping close to 0\% below \gls{snr} $\sim 20$, as some noise samples are estimated to have \gls{snr} $\sim 20$. Furthermore, evaluating the second output on 3000 signals from the validation set gives an accuracy of about 95\% and a false positive rate of about 6\%. These numbers don't sound terrible but are in context. First of all, signals are expected to be in the \gls{snr}-range of 5-15 \textcolor{red}{[Citation]}, secondly, a false positive rate of 6\% equates to a false alarm rate of about \SI[per-mode=fraction]{6e5}{\samples\per\month}\footnote{This crude calculation uses 0.5 as a threshold value. Therefore, one gets $6\times 10^5$ samples with the output being larger than 0.5 per month.}. A month for this work is defined to be \SI{30}{\days}.\medskip\\
From this point the first major iteration was the use of inception modules \cite{inception_module} instead of simple convolution layers (see 15.02.2019 and 03.04.2019 in \cite{network_wiki}). The original implementation did not help

\subsection{Final Network}
\textcolor{blue}{Talk about how the final network looks, how it performs and what could be imporved.}
\subsubsection{Architecture}
\textcolor{blue}{Explain the architecture and the reasoning behind it. Talk about the size of the model, where it could be trained and what the drawbacks of the architecture are. (Drawbacks: Large memory size (hence small batch size), very deep $\to$ slow training and maybe vanishing gradients, hyper-parameter-optimization is hard, not everywhere are residual connections (fixable by further dimensional reduction))}
\subsubsection{Network Performance}
\textcolor{blue}{Evaluate the performance of the network. Show sensitivity curves, talk about speed advantages, how does it in both cases compare to matched filtering? How does it compare to related works? (Reference the BNS-Net paper, what is different between our approach and theirs? Why does theirs seem to work a lot better? Does it?)}
%\textcolor{blue}{Compare the speeds of both methods, the accuracy. What kind of drawbacks does the network have, what are its advantages, how should it be used and understood? (Not as a standalone method of analysis but rather as a starting point for samplers and a quick way of estimate certain properties.)}
