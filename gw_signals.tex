\section{Gravitational-Wave Signals from Binary Neutron Star Mergers}\label{sec:gravitational_waves}
%\textcolor{blue}{Explain the use for this section.}\\
Gravitational waves from two inspiraling neutron stars are among the most interesting signals gravitational wave detectors can detect. They convey information about the highly relativistic regimes of gravity, about the structure of the component stars and about the formation channels of black holes or heavy neutron stars. \textcolor{red}{[Citations]} They are however also very hard to detect, as binary neutron star (\gls{bns}) systems are very light, when compared to inspiraling binary black holes (\gls{bbh}).\\
Part 1 of this section will discuss how gravitational waves (\gls{gw}) are formed and what influences the structure of the resulting waveforms. Part 2 will go over the current method of detecting \gls{gw} and discuss the advantages and drawbacks. \textcolor{red}{Need to specify, that I use Einstein sum convention in this section and that latin indices are spacial indices, whereas greek indices are over all four components. NEED TO CHANGE MOST CITATIONS FROM BACHELOR THESIS TO ORIGINAL SOURCES! Mention bachelor thesis only as a way to look up detailed calculations. Need to specify which convention is used for $\eta^\mn$. Look for ''energy-momentum-tensor'' and replace by ''energy-momentum tensor''. Search for ''chirp-mass'' and replace by ''chirp mass''. Through entire work replace ''decent'' with ''descent'' if some value goes down. (The other just means it was okay)}
\subsection{The Waveform}\label{sec:the_waveform}
\textcolor{blue}{Explain how the waveform looks like, what it depends on, maybe give the concept how it works in the context of linearized theory (quote bachelor thesis), cite important papers regarding the waveform theory.}\\
\textcolor{red}{Maybe mention here that we only look at inspiral due to frequency range, but merger and ringdown also exist.}\\
A gravitational wave generally consists of three stages, an inspiral, a merger and a ringdown phase. In the first phase, the two bodies spiral slowly towards each other following quasi-circular orbits. Once they are close enough though, they will not be able to stay on these circular paths and plunge towards each other. This plunge results in the two bodies merging. Therefore, this phase is commonly referred to as ''merger''. After they collided the remnant will radiate off some more energy. This is the so-called ringdown phase.\\
We will see that frequencies of \gls{gw}s coming from a \gls{bns} system are very high due to their low total mass. As such even the inspiral phase will contain frequencies higher than the ones the network can detect. For this reason we only model the inspiral phase in the following subsections.
\subsubsection{Linearized Gravity}\label{sec:linearized_gravity}
Most calculations of this section are done in detail in \cite{bachelor}.\\
Gravitational waves are a prediction of the Einstein-equation
\begin{equation}\label{def:einstein_equation}
\mathcal{G}_\mn = \frac{8\pi G}{c^4}T_\mn,
\end{equation}
if a weak field is assumed. Here $\mathcal{G}_\mn$ is the Einstein-tensor, $T_\mn$ is the energy-momentum tensor, $G$ is the gravitational constant and $c$ is the speed of light in vacuum. The weak field limit is given by
\begin{equation}\label{def:linear_approximation}
g_\mn = \eta_\mn + h_\mn,
\end{equation}
where $g_\mn$ is the metric of curved space time, $\eta_\mn=\text{diag}\lr{-,+,+,+}$ is the metric of flat space time and $h_\mn$ is a small perturbation, with $\left|h_\mn\right|\ll 1$ and $\left|\partial_{\sigma_1}\dotsc\partial_{\sigma_n}h_\mn\right|\in\mathcal{O}\lr{\left|h_\mn\right|}\eqqcolon\mathcal{O}\lr{h}$. With this approximation the Einstein-equation \eqref{def:einstein_equation} simplifies to
\begin{equation}\label{def:einstein_linear}
\mathcal{G}_\mn=\frac{1}{2}\lr{\partial_{\alpha\mu}h^\alpha_\nu+\partial^\alpha_\nu h_{\mu\alpha} - \partial_\mn h- \Box h_\mn - \eta_\mn\Box h} = \frac{8\pi G}{c^4}T_\mn,
\end{equation}
where $h\coloneqq\eta^\mn h_\mn$ and $\Box\coloneqq\eta^\mn\partial_\mn$.\\
This equation has $10$ independent components of which only $2$ are physical. To reduce the number of independent components, one can choose gauge conditions through the coordinate transformation ${x'}^\mu=x^\mu+\xi^\mu$, which leaves the Einstein equation invariant. One of these gauge conditions is the DeDonder gauge
\begin{equation}
\partial^\alpha \bar{h}_{\alpha\mu} = 0,
\end{equation}
where $\bar{h}_\mn\coloneqq h_\mn - \frac{1}{2}\eta_\mn h$. It can be realized by choosing $\Box\xi_\mu =\partial^\alpha \bar{h}_{\alpha\mu}$. In this gauge the linearized Einstein equation \eqref{def:einstein_linear} reduces to
\begin{equation}\label{def:gw_equation}
\Box\bar{h}_\mn=-\frac{16\pi G}{c^4}T_\mn.
\end{equation}
This gauge however doesn't fix $h_\mn$ completely, as another transformation ${x'}^\mu=x^\mu+\xi^\mu$ could be applied when $\Box\xi_\mu =0$. This can be used in a way that $\bar{h}=-h=0$ and $\bar{h}_{0\mu} = 0 = \bar{h}_{3\mu}$ are also satisfied. The gauge is named transverse-traceless-gauge (\gls{tt}) and results in the metric to be of the form
\begin{equation}\label{def:tt_gauge}
h_\mn^\text{\gls{tt}}=
\begin{pmatrix}
	0 & 0         & 0        & 0\\
	0 & h_+       & h_\times & 0\\
	0 & h_\times & -h_+      & 0\\
	0 & 0         & 0        & 0\\
\end{pmatrix}.
\end{equation}
\eqref{def:tt_gauge} now has only the two independent components $h_+$ and $h_\times$ left, which are called the ''plus-'' and ''cross-polarization'' of a \gls{gw}.\medskip\\
Evaluating \eqref{def:gw_equation} in vacuum reveals the wave-like character of $h_\mn$, as
\begin{equation}\label{def:gw_wave_equation}
\Box\bar{h}_\mn =0
\end{equation}
is a wave equation. Its solutions travel at the speed of light. Therefore, gravitational waves travel through space-time at the speed of light. The effect that a solution of this equation has on a ring of resting test masses is shown in \autoref{fig:gw_test_masses}. (chapter 3 \cite{bachelor})\medskip\\
\begin{figure}
\centering
\includegraphics[width=\textwidth]{effect_gw.pdf}
\caption[Effect of GW on ring of test masses]{This image is taken from \cite{bachelor}. It shows the effect of a \gls{gw} passing orthogonally through a ring of test masses.}\label{fig:gw_test_masses}
\end{figure}
\eqref{def:gw_wave_equation} shows that \gls{gw} exist and can travel through space. It does, however, not specify how these waves are produced. To do so the energy-momentum-tensor cannot be set to $0$. Instead the full equation \eqref{def:gw_equation} needs to be solved. The solution is known to be
\begin{equation}
\bar{h}\lr{t, \vec{x}}=\frac{4 G}{c^4}\int\Diff{3}x'\ \frac{T_\mn\lr{t-\frac{\norm{\vec{x}-\vec{x}'}}{c},\vec{x}'}}{\left|\vec{x}-\vec{x}'\right|}.
\end{equation}
For simplification, it is assumed that the observer is far away from the source when compared to the size of the support of $T_\mn$, such that $\left|\vec{x}-\vec{x}'\right|\approx r\coloneqq\left|\vec{x}\right|$. Therefore we need to solve
\begin{equation}
\bar{h}\lr{t, \vec{x}}=\frac{4 G}{c^4}\frac{1}{r}\int\Diff{3}x'\ T_\mn\lr{t-\frac{r}{c},\vec{x}'}.
\end{equation}
This equation can be solved to yield
\begin{equation}
h_{ab}^\text{\gls{tt}}\lr{t,\vec{x}}=\frac{2 G}{c^4}\frac{1}{r}\ddot{I}_{ab}^\text{\gls{tt}}\lr{t-r/c},
\end{equation}
where $\ddot{I}_{ab}^\text{\gls{tt}}$ is the transverse-traceless-projection of the second time derivative of the second mass moment
\begin{equation}\label{def:quad_st_1}
\ddot{I}^{ab}=c^2\partial_0^2 \int\Diff{3}x'\ x'^a x'^b T^{00} = 2\int\Diff{3}x'\ T^{ab}.
\end{equation}
As the quadrupole moment $Q_{ab}$ is simply the traceless second mass moment and we project it to its traceless part anyways, \eqref{def:quad_st_1} can be rewritten as
\begin{equation}\label{def:h_quadrupole}
\boxed{h^{\gls{tt}}_{ab}\lr{t,\vec{x}} = \frac{2 G}{c^4}\frac{1}{r}\ddot{Q}^{\gls{tt}}_{ab}\lr{t-r/c}}
\end{equation}
with $Q_{ab}\coloneqq I_{ab} - \frac{1}{3}\delta_{ab}I^c_c$. This is the famous quadrupole formula. (chapter 5.2 in \cite{bachelor})\medskip\\

\noindent To calculate the \gls{gw} a binary system emits, $I_{ab}$ or $Q_{ab}$ needs to be specified. Furthermore, the transverse-traceless-projection needs to be calculated. The projection turns out to be ((3.64) in \cite{gwv1})
\begin{equation}
\ddot{I}_{ab}^\text{\gls{tt}}=
\begin{pmatrix}
\lr{\ddot{I}_{11}-\ddot{I}_{22}}/2 & \ddot{I}_{12}                     & 0\\
\ddot{I}_{21}                      & -\lr{\ddot{I}_{11}-\ddot{I}_{22}}/2 & 0\\
0                                  & 0                                 & 0
\end{pmatrix}_{ab}.
\end{equation}
Therefore, the waveforms are given by
\begin{align}
h_+ & = \frac{1}{r}\frac{G}{c^4}\lr{\ddot{I}_{11}-\ddot{I}_{22}}\\
h_\times & = \frac{2}{r}\frac{G}{c^4}\ddot{I}_{12}.
\end{align}
\textcolor{red}{Check if inspiral is said somewhere before. Maybe mention once more, that frequency change is is not in effect, we look at a system with given dynamics. Clarify.}\\
The approximations that led to \eqref{def:einstein_linear} restrict the validity of the results above to cases where there are only slight perturbations to flat space-time. \eqref{def:linear_approximation} actually assumes the background to be flat. Therefore, the dynamics of the two bodies orbiting each other are dictated by Newtonian gravity. With this in mind, the binary system we are trying to model is a system of two point-particles with masses $m_1$, $m_2$. For simplicity\footnote{It turns out that this simplification is very accurate. This is due to two reasone. First of all a possibile ellipticity is radiated away before the \gls{gw} reaches currently detectable frequencies (4.1.3 in \cite{gwv1}). Secondly the rate of change of the orbital radius is small in the regime, where the approximation of linear gravity is meaningful. (4.1.1 in \cite{gwv1})} assume circular motion. In Newtonian mechanics, this problem reduces to an effective one body problem with the reduced mass $\mu=\frac{m_1m_2}{m_1+m_2}$. The motion in these relative coordinates is given by
\begin{equation}\label{def:binary_motion}
\vec{r}\lr{t} = R\cdot
\begin{pmatrix}
	-\sin\lr{\omega_s t}\\
	\cos\lr{\omega_s t}\\
	0
\end{pmatrix},
\end{equation}
where $R$ is the orbital separation of the two point masses and $\omega_s$ the orbital frequency. As a result one gets
\begin{equation}\label{def:binary_system_second_mass_moment}
\left[ I^{ab}\right] = \mu R^2\begin{pmatrix}
		\sin^2\lr{\omega_s t} & -\frac{1}{2}\sin\lr{2\omega_s t} & 0\\
		-\frac{1}{2}\sin\lr{2\omega_s t} & \cos^2\lr{\omega_s t} & 0\\
		0 & 0 & 0
	\end{pmatrix}
\end{equation}
and subsequently
\begin{equation}\label{def:binary_system_second_mass_moment_time_derivative}
\left[ \ddot{I}^{ab}\right] = 2 \mu R^2\omega_s^2
	\begin{pmatrix}
		\cos\lr{2\omega_s t} & \sin\lr{2\omega_s t} & 0\\
		\sin\lr{2\omega_s t} & -\cos\lr{2\omega_s t} & 0\\
		0 & 0 & 0
	\end{pmatrix}.
\end{equation}
Therefore, the amplitudes are given by
\begin{align}\label{def:gw_source_frame}
h_+ &= \frac{4}{r}\frac{G}{c^4}\mu R^2 \omega_s^2\cos\lr{2\omega_s t}\nonumber\\
h_\times &= \frac{4}{r}\frac{G}{c^4}\mu R^2 \omega_s^2\sin\lr{2\omega_s t}.
\end{align}
Interestingly, the frequency of the \gls{gw} is twice the frequency of the orbital period.\\
The equations \eqref{def:gw_source_frame} are written in the source frame, i.e. they are the \gls{gw}-polarizations emitted in the z-direction, where the z-axis is the one orthogonal to the orbital plane and at the center of mass. When measuring these waves we are not assured that the system emits face-on to our detectors. Therefore, we need to change coordinates to get the emission in a general direction $\hat{n}$. To do so, one simply has to transform the second mass moment into the new frame. These calculations can be found on p.111 in \cite{gwv1} which finally yield
\begin{align}\label{def:gw_travel_direction}
h_+ &= \frac{4}{r}\frac{G}{c^4}\mu R^2 \omega_s^2\lr{\frac{1+\cos^2\lr{\iota}}{2}}\cos\lr{2\omega_s t+2\Phi}\nonumber\\
h_\times &= \frac{4}{r}\frac{G}{c^4}\mu R^2 \omega_s^2\cos\lr{\iota}\sin\lr{2\omega_s t+2\Phi},
\end{align}
where $\iota$ is the inclination and $\Phi$ is the phase of the wave at $t=0$.\medskip\\
To be measured these waves need to hit a detector. The LIGO and Virgo detectors are advanced Michelson interferometers with an angle of $\pi/2$ between the two arms. If the \gls{gw} hits such a detector, it will cause a deviation $\delta l$ in arm lengths given by the detector response functions
%\delta l = F_+\lr{\theta, \varphi} h_+ + F_\times\lr{\theta, \varphi} h_\times,
\begin{equation}\label{def:detector_response}
\delta l = F_+\lr{\theta, \varphi}\lr{\cos\lr{2\psi}h_+ - \sin\lr{2\psi} h_\times} + F_\times\lr{\theta, \varphi}\lr{\sin\lr{2\psi}h_+ + \cos\lr{2\psi} h_\times},
\end{equation}
with $h_+$ and $h_\times$ as given in \eqref{def:gw_travel_direction} and
\begin{align}
F_+\lr{\theta, \varphi} &\coloneqq \frac{1}{2}\lr{1+\cos^2\lr{\theta}}\cos\lr{2\varphi}\nonumber\\
F_\times\lr{\theta, \varphi} &\coloneqq\cos\lr{\theta}\sin\lr{2\varphi}.
\end{align}
These response functions ignore the relative motion between the earth and the emitting system, as signals from binary systems spend only a couple seconds within the sensitive frequency band. \cite{gw170817}\\
The angle $\theta$ is taken between the propagation direction of the \gls{gw} to the (outwards facing, relative to earth) normal of the detector. $\varphi$ is the angle between one arm of the detector\footnote{If the arms were labeled with $x$ and $y$ in such a way that they form a right handed coordinate system with the outwards facing normal vector, the arm the angle $\varphi$ is taken to is the one labeled $x$.} to the projection of the propagation direction of the \gls{gw} into the detector-plane. Therefore, the angles $\theta$ and $\varphi$, or rather their projection onto a global coordinate system, are the declination and right ascension respectively. The angle $\psi$ is known as the polarization angle and is not detectable for a single detector. This is due to the reason that rotating the wave around its propagation axis has the same effect.\medskip\\
All the calculations above disregarded the energy carried away by the \gls{gw}. To include it one needs to calculate the luminosity of a \gls{gw}-source, which in turn requires the computation of an effective energy-momentum tensor of the \gls{gw} itself.\\
To get this energy-momentum tensor, second order corrections in $h$ of $R_\mn$ need to be computed and averaged over time. The result is \textcolor{red}{[Citation]}
\begin{equation}
t_\mn = \frac{c^4}{32\pi G}\langle\partial_\mu h^{\sigma\alpha}\partial_\nu h_{\sigma\alpha}\rangle.
\end{equation}
The luminosity is the energy flux at spatial infinity and thus given by
\begin{equation}
L_\text{\gls{gw}} = \lim_{r\to\infty}\int_{S^2\lr{r}}\diff\vec{n}\ \vec{S},
\end{equation}
where $S^i=-c\cdot t^{0i}$ and $S^2\lr{r}$ denotes the spherical shell of radius $r$. When solving this integral and using \eqref{def:h_quadrupole} one gets
\begin{equation}\label{def:luminosity_quadrupole}
\boxed{L_\text{\gls{gw}}=\frac{G}{5c^5}\langle\dddot{Q}^{ab}\dddot{Q}_{ab}\rangle.}
\end{equation}
This equation can now be applied to the binary system specified by \eqref{def:binary_system_second_mass_moment}. To simplify notation and to give measurable parameters, notice that the dynamics of the system under consideration are governed by Newtonian physics and thus Kepler's laws apply. Especially Kepler's third law
\begin{equation}\label{def:kepler_third_law}
\omega_s^2=\frac{G M}{R^3}
\end{equation}
will be of use, where $M=m_1+m_2$ is the total mass of the system. Using \eqref{def:kepler_third_law} to eliminate $R$ in \eqref{def:binary_system_second_mass_moment} and inserting this equation into \eqref{def:luminosity_quadrupole} yields
\begin{equation}\label{def:luminoisity_binary_system}
L_\text{\gls{gw}}=\frac{32}{5}\frac{c^5}{G}\lr{\frac{G\omega_s M_c}{c^3}}^{10/3},
\end{equation}
where
\begin{equation}
M_c\coloneqq \mu^{3/5}M^{2/5}=\frac{\lr{m_1 m_2}^{3/5}}{\lr{m_1 + m_2}^{1/5}}.
\end{equation}
$M_c$ is called the chirp mass and is the only mass-combination a \gls{gw} depends on in linearized theory.\medskip\\
According to \eqref{def:luminoisity_binary_system}, a binary system looses energy when emitting \gls{gw}. The energy that is carried away is taken from the orbital energy $E_\text{orbit}$ of the binary system. Therefore, disregarding any other effects\footnote{These effects could for instance be tidal deformation, mass acquisition or other sources of gravity in the proximity of the binary system.} that might cause $E_\text{orbit}$ to vary, we get
\begin{equation}
-\frac{d E_\text{orbit}}{dt}=-\frac{1}{2}\frac{G m_1 m_2 \dot{R}}{R^2}\mbe L_\text{\gls{gw}}.
\end{equation}
One can again utilize \eqref{def:kepler_third_law} to eliminate $R$ and $\dot{R}$ in favor of $\omega_s$ and $\dot{\omega}_s$. Furthermore, $\omega_s = \pi f_\text{\gls{gw}}$, and thus
\begin{equation}\label{def:frequency_evolution_differential}
\dot{f}_\text{\gls{gw}}=\frac{96}{5}\pi^{8/3}\lr{\frac{G M_c}{c^3}}^{5/3}f_\text{\gls{gw}}^{11/3}.
\end{equation}
This equation describes a runaway process, as for a positive value $f_\text{\gls{gw}}$ the change in frequency is positive, leading to a larger value of $f_\text{\gls{gw}}$ and so on. This in turn by \eqref{def:kepler_third_law} means that the two masses will come closer and closer together until they touch. The point in time at which the waveform shuts off, will be denoted by $t_\text{coal}$. With this, one can define the time until coalescence $\tau=t_\text{coal}-t$ and solve the differential equation \eqref{def:frequency_evolution_differential}. \textcolor{red}{[Cite p.170 \cite{gwv1}]}
\begin{equation}\label{def:frequency evolution}
f_\text{\gls{gw}}\lr{\tau} = \frac{1}{\pi}\lr{\frac{5}{256}\frac{1}{\tau}}^{3/8}\lr{\frac{G M_c}{c^3}}^{-5/8}
\end{equation}
Now that the frequency evolution of a \gls{gw} is known, the amplitudes $h_+$ and $h_\times$ can also be modeled. To do so, revisit the initial assumption \eqref{def:binary_motion}. In this equation $R$ will now be time dependent and $\omega_s t$ will be replaced by $\Phi\lr{t}$, where
\begin{equation}\label{def:phase_equation}
\Phi\lr{t} = 2\pi \int_{t_0}^t\diff t'\ f_\text{\gls{gw}}\lr{t'}.
\end{equation}
In principle, all time derivatives in \eqref{def:binary_system_second_mass_moment_time_derivative} would need to be redone, taking into account the time dependence of $\omega_s$ and $R$. However, the approximations that have led to these waveforms are quite strong. The rates of change $\dot{\omega}_s$ and $\dot{R}$ will only have non-negligible contributions when frequencies are pretty high and the orbital separation $R$ is small. In these regimes the linear approximation \eqref{def:linear_approximation} will be invalid. Therefore, we can neglect the contributions of $\dot{\omega}_s$ and $\dot{R}$ and still get a qualitative look into the dynamics of the system. Hence, replace $\omega_s$ in the prefactor of \eqref{def:gw_travel_direction} by $\pi f_\text{\gls{gw}}\lr{t}$, $2\omega_s t + 2 \Phi$ by $\Phi\lr{t}$ and $R$ by \eqref{def:kepler_third_law}.\\
With \eqref{def:frequency evolution}, equation \eqref{def:phase_equation} can be solved to yield
\begin{equation}
\Phi\lr{\tau}=-2\lr{\frac{5 G M_c}{c^3}}^{-5/8}\tau^{5/8}+\Phi_0,
\end{equation}
where $\Phi_0$ is the phase at $\tau=0$, i.e. at coalescence. Therefore, this value is called the coalescence phase. Combining these results, one gets the time dependent waveforms
\begin{align}\label{def:linear_waveforms_time_evolution}
h_+\lr{t} & = \frac{1}{r}\lr{\frac{G M_c}{c^2}}^{5/4}\lr{\frac{5}{c\tau}}^{1/4}\lr{\frac{1+\cos^2\lr{\iota}}{2}}\cos\lr{\Phi\lr{\tau}}\nonumber\\
h_\times\lr{t} & = \frac{1}{r}\lr{\frac{G M_c}{c^2}}^{5/4}\lr{\frac{5}{c\tau}}^{1/4}\cos\lr{\iota}\cos\lr{\Phi\lr{\tau}}.
\end{align}
Inserting \eqref{def:linear_waveforms_time_evolution} and \eqref{def:detector_response} shows that in linearized theory the output of a detector depends on 8 parameters. These are the luminosity distance $r$, the chirp mass $M_c$, the coalescence time $t_\text{coal}$, the coalescence phase $\Phi_0$, the inclination $\iota$, the declination $\theta$, the right ascension $\varphi$ and the polarization angle $\psi$. The first four parameters are source intrinsic parameters, where $M_c$ is a combination of the component masses $m_1$ and $m_2$. In that sense the parameter space can be extended to be 9-dimensional.\\
\autoref{fig:linear_waveform} shows an example of the time evolution of a waveform described by \eqref{def:linear_waveforms_time_evolution}. \textcolor{red}{To obtain the spin effects in linearized gravity, one could write down the lagrangian of two spinning particles orbiting each other, solving the lagrange equation for the trajectories of the particles and insert these trajectories into the definition of the quadrupole tensor. (At least that's how I think it could be done. If there is time, maybe do these calculations.)}\\
Alongside energy, \gls{gw} also carry away angular momentum from the source. Using the quadrupole radiation \eqref{def:h_quadrupole}, the change in angular momentum evaluates to ((3.97) in \cite{gwv1})
\begin{equation}
\frac{dJ^i}{dt}=\frac{2G}{c^5}\varepsilon^{ikl}\langle \ddot{Q}_{ka}\dddot{Q}_{la}\rangle .
\end{equation}
The angular momentum that is carried away comes from the total angular momentum of the source. This in turn is comprised of the orbital angular momentum as well as the individual spins of the two component masses of a binary system. Therefore it is at least qualitatively understandable that the spins of the two bodies has an effect on the evolution of the waveform. Thus the 9 parameters of a waveform can be extended to 15, if both objects are allowed to rotate. The 6 additional parameters are the individual angular momenta of the two masses. Two further parameters influence the waveform if the objects are not rigid and allowed to deform. This is true for binary neutron stars, as neutron stars are not singularities.\\
Even though this work deals with \gls{bns}-signals, we neglect spin effects and tidal deformability and will thus not go into further detail here. \textcolor{red}{For more information on spin- and tidal effects see [Citations].}\\
\begin{figure}
\centering
\includegraphics[width=\textwidth]{linear_waveform.png}
\caption[Example time-evolution of a linear waveform]{This figure is derived from Figure 4.1 in \cite{gwv1}. Shown is an example of a waveform as it could be observed by a detector if the linear waveforms describe the source accurately. Note the frequency and amplitude evolution, they both rise simultaneously. This behavior is called ''chirping''.}\label{fig:linear_waveform}
\end{figure}

\subsubsection{Post-Newtonian Expansion}\label{sec:pn_expansion}
This part closely follows chapter 5 in \cite{gwv1}, mainly stating results and concepts.\\
As we will see in \autoref{sec:matched_filtering}, accurate models for the waveforms are necessary to detect \gls{gw} using traditional methods. This does not change for an approach utilizing machine learning algorithms, as they can only detect waveforms from distributions they have sampled before. \textcolor{red}{Citation?} Though the waveform derived in \eqref{def:linear_waveforms_time_evolution} gives a good qualitative overview of the rough amplitude evolution of inspiraling binary systems, it is hardly an accurate model. The main drawback of this model are the dynamics used to describe the motion of the system. It assumes equation \eqref{def:linear_approximation} which in turn means that \gls{gws} and source-dynamics can be separated. Therefore, in linearized gravity the motion of a binary system is dictated by Newtonian dynamics, while the \gls{gws} are a relativistic effect.\\
This issue is overcome by taking an approach called post-Newtonian expansion (\gls{pn}-expansion). To approximate the solution of the full Einstein equation \eqref{def:einstein_equation}, $g_\mn$ is expanded in powers of the small parameter $\epsilon\sim v/c \sim \lr{R_s / d}^{1/2}$, instead of using $g_\mn\approx h_\mn +\eta_\mn$. Here $v$ is the typical speed inside the source, $R_s$ is the Schwarzschild radius attributed to the system's total mass and $d$ is the diameter of a world-tube containing the support of the energy-momentum-tensor of the source. Therefore, $\epsilon$ is small if the source is not too compact and speeds are low compared to the speed of light.\\
Specifically the metric expands to \textcolor{red}{(Mention why they start at different orders?)}
\begin{align}\label{def:pn_expansion_metric}
g_{00} && = & -1 && +{g^{(2)}}_{00} & +{g^{(4)}}_{00} && +{g^{(6)}}_{00} && +\dotsc\nonumber\\
g_{0j} && = & && & +{g^{(3)}}_{0j} && +{g^{(5)}}_{0j} && +\dotsc\\
g_{ij} && = & &&\delta_{ij} & +{g^{(2)}}_{ij} && +{g^{(4)}}_{ij} && +\dotsc,\nonumber
\end{align}
where ${g^{(n)}}$ denotes a term $\sim \epsilon^n$. The energy-momentum-tensor can be expanded in an equivalent way
\begin{align}\label{def:pn_expansion_energy_momentum_tensor}
T^{00} & = {T^{(0)}}^{00}+{T^{(2)}}^{00}+\cdots\nonumber\\
T^{0i} & = {T^{(1)}}^{0i}+{T^{(3)}}^{0i}+\cdots\\
T^{ij} & = {T^{(4)}}^{ij}+{T^{(4)}}^{ij}+\cdots.\nonumber
\end{align}
These expressions than need to be inserted into \eqref{def:einstein_equation} and terms of the same order in $\epsilon$ need to be equated. To get the equations of motion to the $n$-th \gls{pn}-order terms up to order $\epsilon^{2n}$ need to be kept and computed. Therefore it is possible to have the correction of some quantity to \gls{pn}-order $2.5$.\\
To get the 1-\gls{pn} order corrections to the metric, one can once again impose a gauge condition. \textcolor{red}{Gauge condition can be required for any order.} The gauge condition most commonly used is still called deDonder gauge, which in the \gls{pn}-case reads
\begin{equation}
\partial_\mu\lr{\sqrt{-g}g^\mn}=0,
\end{equation}
where $g$ is the determinant of $g_\mn$. In this gauge the Einstein equation yields (to 1-\gls{pn} order)
\begin{align}\label{def:metric_differential_equation_pn_1}
\Delta {g^{(2)}}_{00} = & -\frac{8\pi G}{c^4}{T^{(0)}}^{00}\nonumber\\
\Delta {g^{(2)}}_{ij} = & -\frac{8\pi G}{c^4}\delta_{ij}{T^{(0)}}^{00}\nonumber\\
\Delta {g^{(3)}}_{0i} = & \frac{16\pi G}{c^4}{T^{(1)}}^{0i}\\
\Delta {g^{(4)}}_{00} = & \partial_0^2{g^{(2)}}_{00} + {g^{(2)}}_{ij}\partial_i\partial_j{g^{(2)}}_{00}-\partial_i{g^{(i)}}_{00}\partial_j{g^{(2)}}_{00}\nonumber\\
& -\frac{8\pi G}{c^4}\left\{{T^{(2)}}^{00}+{T^{(2)}}^{ii}-2{g^{(2)}}_{00}{T^{(0)}}{00}\right\},\nonumber
\end{align}
with $\Delta = \delta^{ij}\partial_i\partial_j$. Observe that higher order terms in \eqref{def:metric_differential_equation_pn_1} depend on the lower order terms of the expansion. Therefore the \gls{pn}-expansion is an iterative process.\\
The equations \eqref{def:metric_differential_equation_pn_1} do in principle have many solutions. A particular solution, however, is specified by the boundary condition. A typical boundary condition is the ''no incoming radiation'' condition, where it is required that the metric approaches the flat space time metric $\eta$ as one goes to spatial infinity. The \gls{pn}-expansion however is only valid in the near region of the source, as it approximates the retarded solutions by a series of instantaneous potentials. To use a correct boundary condition one approximates the far-field solution and matches it with the near-field \gls{pn}-solution.\\
The approximation for the far-field is called Post-Minkowskian-expansion (\gls{pm}-expansion). For simplified notation the Einstein equation will be recast into its relaxed form
\begin{equation}\label{def:einstein_equation_relaxed}
\Box k^\mn = \frac{16\pi G}{c^4}\tau^\mn.
\end{equation}
To get this result, the deDonder gauge
\begin{equation}
\partial_\nu k^{\mn}=0
\end{equation}
was again required. The metric $k$ is given by
\begin{equation}
k^\mn\coloneqq \sqrt{-g}g^\mn -\eta^\mn.
\end{equation}
Furthermore
\begin{equation}\label{def:pn_tau}
\tau^\mn = \lr{-g}T^\mn + \frac{c^4}{16\pi G}\Lambda^\mn,
\end{equation}
with $\Lambda^\mn$ being a tensor that depends highly nonlinearly on $k$ and $g$. The expression is given in (5.74) of \cite{gwv1}. To simplify the relaxed Einstein equation \eqref{def:einstein_equation_relaxed} in the far field, we denote that the energy-momentum-tensor of matter $T^\mn$ vanishes outside the source. Therefore the task is solving
\begin{equation}
\Box k^\mn = \Lambda^\mn.
\end{equation}
To do so, we expand $k$ in powers of $R_s/r$, which is equivalent to expanding in powers of $G$. We also expand $\Lambda^\mn$ in powers of $k$. Therefore
\begin{gather}
k^\mn=\sum_{n=1}^\infty G^n k^\mn_n\\
\Lambda^\mn = N^\mn\left[k,k\right] + M^\mn\left[k,k,k\right]+\dotsc,
\end{gather}
where $N^\mn\left[ k, k\right]$ denotes a tensor of quadratic order in $G$. Equating terms of the same order in $G$ and iteratively using the results for $k$ yields
\begin{gather}
\Box k^\mn_1=0\\
\Box k^\mn_2 = N^\mn\left[k_1, k_1\right]\\
\Box k^\mn_3 = M^\mn\left[k_1, k_1, k_1\right]+N^\mn\left[k_1,k_2\right]+N^\mn\left[k_2,k_1\right]\\
\vdots\nonumber
\end{gather}
or in short
\begin{equation}\label{def:pm_wave_equation}
\Box k^\mn_n=\Lambda^\mn_n\left[k_1, \dotsc, k_{n-1}\right].
\end{equation}
The most general solution to $k_1$ can be written in terms of retarded multipolar waves. The solutions under consideration of the deDonder gauge can be found in (5.95) and the following equations of \cite{gwv1}. They consist of multiple retarded potentials and form a multipole expansion of $k_1$. To find the solution to any order $n$ equation \eqref{def:pm_wave_equation} needs to be solved, inserting all previous solution $k_1,\dotsc,k_{n-1}$. Therefore, a general solution to \eqref{def:pm_wave_equation} would be convenient.\\
Traditionally such a solution is known and given by the retarded Green's function. The problem, however, is that the solution to \eqref{def:pm_wave_equation} is only valid outside the source but solving it by the retarded Green's function requires knowledge over the entire region. To get around this issue, we observe the fact that we only want the solution of $k$ to some finite order in $G$. This has the benefit that only a finite number of multipole terms of $\Lambda^\mn_n$ have to be used. The finite number of terms enables us to find some constant $B$, such that $r^B\Lambda^\mn_n$ is defined for all $r$ and thus its solution is given by the retarded Green's function. For $B\to 0$ the original divergence is recovered and the multipole expansion has poles. Therefore, near $B=0$ the solution $I^\mn_n\lr{B}=\Box^{-1}_\text{ret}\lr{r^B\Lambda^\mn_n}$ can be expanded in a Laurent-series. Taking only the zeroth order term yields a particular solution $u^\mn_n$ with
\begin{equation}
\Box u^\mn_n=\Lambda^\mn_n.
\end{equation}
From this particular solution the general solution can be constructed by adding the homogeneous solution.\medskip\\
As a final step, the \gls{pn}-expansion can be recast in the form of $k$, where we expand
\begin{gather}
k_\mn =\sum_{n=2}^\infty \frac{1}{c^n}\frac{d^n}{du^n}k_\mn\lr{u}\\
\tau^\mn=\sum_{n=2}^\infty \frac{1}{c^n}\frac{d^n}{du^n}\tau^\mn\lr{u},
\end{gather}
with $u=t-r/c$ and $\tau$ given in \eqref{def:pn_tau}. Doing so and inserting it into the relaxed Einstein equations results in the recursive relation
\begin{equation}\label{def:k_solution_pn}
\Delta\left[\frac{d^n}{du^n}k^\mn\right]=16\pi G\frac{d^{n-4}}{du^{n-4}}\tau^\mn+\partial_t^2\left[\frac{d^{n-2}}{du^{n-2}}k^\mn\right].
\end{equation}
Taking the solution for the 1\gls{pn} case discussed above as a starting point the \eqref{def:k_solution_pn} can be solved in a similar fashion to \eqref{def:pm_wave_equation} using only a finite number of terms in a multipole expansion of the potentials.\medskip\\
With this setup we mention once more that the regions of validity for the \gls{pn}- and \gls{pm} expansion overlap but neither are completely solved. To obtain the full solution the \gls{pn}-equations need a boundary condition, whereas the \gls{pm}-equations need a source of some form. Therefore the \gls{pm}-equations can be viewed as the limiting case of the \gls{pn}-case and thus provide a boundary condition. The \gls{pn}-equations on the other hand have fixed multipole potentials that depend on the energy-momentum tensor of matter. These can in turn be used to fix the multipole potentials in the \gls{pm}-equations and one obtains a full solution.\\
At this point we will not go further into further details of the formalism itself but rather look at its influence on the waveforms. \textcolor{red}{For further reference on the PN-formalism see [Citations]. For information about the multipole expansion see [Citations].}\medskip\\
Surprisingly, to 1\gls{pn} order the results are identical to the ones obtained using linearized gravity in \autoref{sec:linearized_gravity}. For convenience one defines the dimensionless quantity
\begin{equation}
x\coloneqq \lr{\frac{G M \omega_s}{c^3}}^{2/3},
\end{equation}
where $M$ is the total mass and $\omega_s$ the orbital frequency. Note that $x\sim \frac{v^2}{c^2}$ and thus the \gls{pn}-expansion can be given in terms of powers in $x$. For further notational simplicity define the symmetric mass ratio
\begin{equation}
\nu\coloneqq \frac{m_1 m_2}{\lr{m_1 + m_2}^2},
\end{equation}
and the post-Newtonian parameter
\begin{equation}
\gamma\coloneqq \frac{G M}{r c^2}.
\end{equation}
Using the metrics acquired from the \gls{pn}-expansion one can solve the equations of motion for the two inspiraling bodies and obtain corrections to $\omega_s$, $\gamma$, the energy $E$ and the radiated power $L_\text{\gls{gw}}$. Combining these results one can than find the phase evolution and emitted waveforms. The computations are extremely long. For this reason we only state the results for the energy and luminosity at $3.5$\gls{pn} order here. ($\mu$ in this case is the reduced mass) (equation (5.256) in \cite{gwv1})
\begin{align}
E= & -\frac{\mu c^2 x}{2}\left\{ 1+\lr{-\frac{3}{4}-\frac{1}{12}\nu}x+\lr{-\frac{27}{8}+\frac{19}{8}\nu-\frac{1}{24}\nu^2}x^2\right.\nonumber\\
& +\left. \left[ -\frac{675}{64}+\lr{\frac{34445}{576}-\frac{205}{96}\pi^2}\nu -\frac{155}{96}\nu^2 - \frac{35}{5184}\nu^3\right] x^3\right\}+\mathcal{O}\lr{\frac{1}{c^8}}
\end{align}
In the equation below $C$ is the Euler-Mascheroni constant. (equation (5.257) in \cite{gwv1})
\begin{align}
L_\text{\gls{gw}} = & \frac{32 c^5}{5 G}\nu^2 x^5 \left\{ 1 + \lr{-\frac{1247}{336}-\frac{35}{12}\nu}x + 4\pi x^{3/2}\right.\nonumber\\
& + \lr{-\frac{44711}{9072}+\frac{9271}{504}\nu+\frac{65}{18}\nu^2}x^2\nonumber\\
& + \lr{-\frac{8191}{672}-\frac{583}{24}\nu}\pi x^{5/2}\nonumber\\
& + \left[ \frac{6643739519}{69854400}+\frac{16}{3}\pi^2 -\frac{1712}{105}C-\frac{856}{105}\log\lr{16 x}\right.\nonumber\\
& + \left. \lr{-\frac{134543}{7776}+\frac{41}{48}\pi^2}\nu - \frac{94403}{3024}\nu^2 - \frac{775}{324}\nu^3\right] x^3\nonumber\\
& + \left. \lr{-\frac{16258}{504}+\frac{214745}{1728}\nu + \frac{193385}{3024}\nu^2}\pi x^{7/2}+\mathcal{O}\lr{\frac{1}{c^8}}\right\}
\end{align}

\subsubsection{TaylorF2}
This section closely follows section 2.4.2 and 2.4.4 of \cite{frank_phd}.\\
The previous section outlines how the energy and luminosity can be obtained to some finite \gls{pn} order. To actually get a waveform from these results, we need to solve
\begin{equation}\label{def:energy_luminosity_pn}
\frac{\diff E}{\diff t}=-L_\text{\gls{gw}}\ \rightarrow\ \frac{\diff v}{\diff t}=-\frac{L_\text{\gls{gw}}}{\diff E / \diff v}
\end{equation}
and from there obtain the phase using
\begin{align}\label{def:phase_pn}
&\Phi = \int \diff t\ \omega,& M\omega=v^3.
\end{align}
There are many ways to approximate solutions to these equations, but we will focus only on the TaylorF2 approximation, as it is the one we use in this work. It starts by inverting equation \eqref{def:energy_luminosity_pn} to get $t(v)$. It therefore tries to solve
\begin{equation}\label{def:pn_equation_t}
\frac{\diff t}{\diff v}=-\frac{\diff E / \diff v}{L_\text{\gls{gw}}}.
\end{equation}
To do so, the right hand side of this equation is re-expanded in terms of $v/c\sim \epsilon$ to $3.5$ \gls{pn} order. This inverted equation can then be analytically integrated and the result can be found in (2.73) of \cite{frank_phd}.\\
Similarly $\Phi$ can be computed from the equations \eqref{def:phase_pn}, using
\begin{equation}\label{def:pn_equation_phi}
\frac{\diff \Phi}{\diff v} = \frac{v^3}{M}\frac{\diff t}{\diff v}=-\frac{\diff E / \diff v}{L_\text{\gls{gw}}}.
\end{equation}
The integral can once more be performed analytically and the result can be found in (2.75) of \cite{frank_phd}.\\
The time domain phase can then be constructed from equations \eqref{def:pn_equation_t} and \eqref{def:pn_equation_phi} and is called TaylorT2. In \autoref{sec:matched_filtering} we will find, however, that the analysis will be done in the frequency domain. One can construct the frequency representation of the orbital phase and in extension the entire waveform from the results of TaylorT2. These waveforms are hence called TaylorF2.
\begin{comment}
Even though the \gls{pn}-formalism allows us to model the waveforms rather accurately for a long part of the signal it still assumes circular motion and velocities to be small. As the two stars spiral together they will, however, reach a point, where circular motion is not possible anymore. This orbit is called the ''innermost stable circular orbit'' (\gls{isco}). From here the stars will plunge towards each other, velocities are high and fields are not weak anymore. Therefore, the validity of the formalism breaks down at this point. Finally, after the two bodies have merged, the remaining body will radiate off some more energy. Therefore, the signal consists of three parts: inspiral, merger and ringdown. The inspiral phase is well modeled by the \gls{pn}-formalism outlined in \autoref{sec:pn_expansion}. The ringdown can be described by \textcolor{red}{What can it be described by and where can one find an analysis of this?} The merger however is very difficult to model and usually involves numerical solutions of the Einstein equations. These, however, are very costly from a computational perspective, which is a great problem when one searches for \gls{gws}.
\textcolor{red}{Transition into roughly describing the PN-approximation and TaylorF2 here.}
%Until now, the approximations made were quite strict. One of the major ones is the approximation of the two bodies of mass $m_1$ and $m_2$ as point particles. As such they have no intrinsic spin. Dropping this restriction shows, that also angular momentum is radiated away. Also each \textcolor{red}{Spin seems to be difficult. If it doesn't fit time wise, consider just pointing out, that there are in theory 6 more parameters coming from the individual spins, but as we only look at non-spinning systems anyways, we will not go into further detail and the effects here. Otherwise maybe look at Stephans Bachelor thesis to get a little grasp on spins and their effects? However I'm not comfortable writing, that spins only come into effect in PN-theory. Maybe I can find a mass momentum $I^{ab}$ for a binary system of two spinning masses and just say, that this can be plugged into the according equations?}\\
\textcolor{red}{Include backreaction in first order as 4.1.1 \cite{gwv1} does, until equation (4.32). Mention that elliptic effects can be mostly disregarded, cite \cite{gwv1}. Go over into PN territory and mention what changes.}
\end{comment}

\subsection{Searching for Gravitational Waves}\label{sec:matched_filtering}
\textcolor{blue}{Explain what matched filtering is, why it works and how it is applied currently. Also need to mention PSD and what it is.}\\
The LIGO Scientific Collaboration and the VIRGO collaboration use a wide range of different online pipelines that aim to rapidly generate triggers if a \gls{gw} event is present within the data \cite{ligo_pipelines}. Most of them are modeled searches and based on the concept of matched filtering, i.e. they need a pre-generated model to find a signal that is similar to this model. As we are using PyCBC to generate our waveforms, backgrounds and results we will be going in detail only about the PyCBC Live pipeline \cite{pycbc_live}. In the first part of this section we will briefly recapitulate the basics of matched filtering before summarizing the PyCBC Live pipeline in the second part.
\subsubsection{Matched Filtering}
\Cref{sec:the_waveform} discussed how \gls{gw} are generated. The detectors register them as a differential arm length of their two interferometer arms. However, the amplitude of this change in distance is on the order \SI[parse-numbers=false]{\mathcal{O}\lr{10^{-18}}}{\meter} and as such is highly contaminated with noise originating from thermal motion of atoms, quantum shot noise and others \cite{frank_phd}. In fact the amplitude output of the detector is dominated by noise (7.3 in \cite{gwv1}). For this section we will assume that the data $s$ contains a signal $h$ submerged in noise $n$. Therefore,
\begin{equation}
s(t) = n(t) + h(t)
\end{equation}
in the time domain. To still be able to detect \gls{gw}s a filtering technique names matched filtering is applied to the data. It is a modeled filter, thus requiring knowledge of the signal it is trying to find. The basic concept goes as follows. Assume we use the exact waveform that is inside the data as a filter on white gaussian noise with zero mean \textcolor{red}{(Do I need these requirements to the noise?)}. Then, to decide whether or not the template is present in the data, we can multiply the template with the data and integrate this product over some time $T$, averaging the output by dividing by $T$.
\begin{equation}\label{def:matched_filter_concept}
\frac{1}{T}\int_0^T\diff t\ s(t)\cdot h(t) = \frac{1}{T}\int_0^T\diff t\ h^2(t) + \frac{1}{T}\int_0^T\diff t\ n(t)\cdot h(t)
\end{equation}
For $T\to\infty$ the first term will approach some finite value $h_0^2$, where $h_0$ is the characteristic amplitude of the \gls{gw}. The second term, though, will approach $0$ as $n(t)$ and $h(t)$ are uncorrelated (7.3 in \cite{gwv1}).\smallskip\\
To understand the optimal filter we can construct from \eqref{def:matched_filter_concept} we firstly need to define the power spectral density (\gls{psd}). The \gls{psd} characterizes the power contained within the noise at a given frequency bin. As $n(t)$ is real valued the fourier transform $\tilde{n}(f)$ fulfills $\tilde{n}(-f)=\tilde{n}^\ast(f)$, where $\ast$ denotes the complex conjugate. The power in each frequency bin is given by
\begin{equation}\label{def:psd}
\langle \tilde{n}^\ast(f)\tilde{n}(f')\rangle = \delta\lr{f-f'}\frac{1}{2}S_n(f),
\end{equation}
where $S_n(f)$ denotes the one sided \gls{psd}. It is called one sided as the factor $1/2$ in \eqref{def:psd} results in
\begin{equation}
\langle n^2(t)\rangle = \int_0^\infty\diff f\ S_n(f).
\end{equation}
Without the factor the integral would have boundaries $\pm\infty$ and need integration over the entire frequency range.\\
The \gls{psd} can be estimated from data by integration over some period of time $T$ and is given by
\begin{equation}\label{def:estimate_psd}
S_n(f)=\frac{2\langle {\left|\tilde{n}(f)\right|}^2\rangle}{T}
\end{equation}
and has the unit $\text{Hz}^{-1}$. If we refer to a \gls{psd} in this work, it will always be the one sided variant. The amplitude spectral density (\gls{asd}) is the square root of the \gls{psd} and used to color noise. Especially it is used to whiten data, i.e. divide the data by the \gls{asd} associated to the \gls{psd} of the underlying noise. Whitening is used to suppress frequencies that are very noisy.\smallskip\\
The matched filter can be derived using the \gls{psd} and is proven to be optimal for gaussian noise. Optimal meaning that the value a template matching the waveform inside the data perfectly is maximized. We use this matched filter to create a statistic called signal-to-noise ratio (\gls{snr}) $\rho$, which is given by \cite{pycbc_live}
\begin{equation}\label{def:snr_matched_filter}
\rho^2\coloneqq\frac{\norm{\langle s|h\rangle}^2}{\langle h|h\rangle},
\end{equation}
with
\begin{equation}
\langle a|b\rangle = 4\int_0^\infty\diff f\ \frac{\tilde{a}(f)\tilde{b}^\ast(f)}{S_n(f)}.
\end{equation}
\textcolor{red}{Maybe say a few words about what the SNR means.}

\subsubsection{Detection Pipeline (PyCBC Live)}
%\textcolor{blue}{Describe how PyCBC Live works and especially go into how they calculate false alarm rates.}\\
This section summarizes the results of \cite{pycbc_live}.\\
PyCBC Live is the online detection pipeline based on the PyCBC library optimized for low latency trigger generation. It utilizes a template bank of $\mathcal{O}\lr{10^5}$ pre-calculated waveforms in a matched filter search. For multiple detectors this analysis is done in all of the detectors and only afterwards combined into a combined statistic. Therefore, the goal is to calculate a \gls{snr} time series for every detector and template. To do so the expression
\begin{equation}
\rho^2(t) = \frac{4}{\langle h|h\rangle}\int_0^\infty\diff f\ \frac{\tilde{s}(f)\tilde{h}^\ast(f)}{S_n(f)}e^{2\pi i f t}
\end{equation}
has to be computed for every template $h$ and all detector data $s$. In order to keep up with the constant data-stream the input is downsampled to \SI{2048}{\hertz} and high-passed. The latter allows for computations to be executed in single precision. Furthermore, the workload can be easily parallelized over multiple compute notes to speed up the process. This process is repeated every \SI{8}{\s} on a chunk of data that contains upwards of \SI{32}{\s} of time series strain data. The \gls{psd} $S_n(f)$ is estimated from the data by dividing the minute prior to the analysis segment into overlapping intervals, each of duration \SI{4}{\s}, and computing equation \eqref{def:estimate_psd} from it. The frequency bins are finally averaged over all intervals.\\
After computing the \gls{snr} time series, the peak \gls{snr} of each one is thresholded by a value of $\sim 5.5$, i.e. no peak \gls{snr} below this value are kept, and all but the loudest $\mathcal{O}\lr{10^3}$ are discarded. Sometimes these triggers are then discarded altogether. This happens if the \gls{psd} estimate shows drifts in detector noise, sensitivity estimates report poor data quality or the \gls{snr} values are unusually high. If non of these conditions apply, some more consistency tests are applied and the \gls{snr} is reweighed based on a $\chi^2$ signal consistency test. Until now this procedure was done for all detectors individually. The resulting single detector triggers are then combined to report a detector network \gls{snr}. To do so, the templates that generated a trigger need to have generated a trigger in all other detectors. The time difference at which the peak was found in each detector is furthermore not allowed to be greater than the time of flight difference, which is \SI{\sim 10}{\milli\s}. The maximum separation is therefore set to \SI{15}{\milli\s}.\\
For the two detectors Hanford and Livingston all remaining trigger pairs are combined to a network \gls{snr} given by
\begin{equation}
\rho_c^2=\rho_\text{H}^2 + \rho_\text{L}^2 + 2\log\lr{p(\theta)},
\end{equation}
where $p(\theta)$ is the astrophysical probability of a trigger observed with parameters $\theta$.\medskip\\
To estimate the significance of the trigger, the false alarm rate of the procedure producing the reported trigger is calculated. In order to get enough data that does not contain a physical signal but captures the current detector performance, the data of one detector is shifted in time. The length of time to shift is determined by the time of flight between the two observatories. If a trigger was found in such time shifted data, we would know that it could not be a real trigger as it violates causality. To do this efficiently only the triggers for each detector are shifted in time. This procedure allows to get false alarm rate estimates of down to 1 per 100 years. The threshold of notifying other astronomers of a trigger is chosen to be 1 per 2 month.\medskip\\
The described procedure produces reliable triggers and through the known optimal template a rough estimate of the skyposition. It is able to keep up with the detector output but introduces on average \SI{16}{\s} of latency, i.e. a trigger is reported \SI{16}{\s} after the event occurred on average.
\begin{comment}
This estimate of the \gls{psd} is also used to detect drifts in the detector noise. Based on this, estimates of poor detector sensitivity and unusually high \gls{snr} values, some data is vetoed and not used for further analysis.
where $h$ is the template. To generate triggers from this time series a threshold of $\sim 5.5$ is applied to the output and several consistency tests are applied. These tests are applied to counteract the influence of non-Gaussian noise sections, \\
\textcolor{red}{Can keep up with live data stream but adds on average 16 seconds of latency, i.e. triggers are generated 16s after the data came in. Pipeline treats every detector on its own and combines the results. Make cuts based on data quality which is estimated by estimating the PSD.}
\end{comment}