\section{Introduction}
\textcolor{Blue}{Text for introduction. State where the source code used in this work can be found, list which software versions were used.}\\
With the first direct detection of a gravitational wave (\gls{gw}) on September the 14th 2015 \cite{gw150914} the age of gravitational wave astronomy began. It opened up the possibilities to test Einstein's theory of gravity in highly relativistic systems \cite{test_gr_gw150914}, sample the population of compact binary systems consisting of objects like neutron stars or black holes \cite{population_binary_systems}, define new astronomical standard candles \cite{standard_candles} and many more. The first and second observation run of the advanced LIGO and Virgo detectors \cite{aligo, avirgo} led to 11 detections of \gls{gw}s from different systems \cite{catalog}. The third observation run, which is currently ongoing, promises to greatly expand this catalog and has already found multiple \gls{gw} candidates.\\
The most promising source of \gls{gw}s are binary systems consisting of two black holes, neutron stars or a mix of these two. So far all confirmed detections of \gls{gw}s were caused by such compact binary systems. Most of them were generated by two coalescing black holes, the only exception being GW170817 \cite{catalog}. The latter, instead, was emitted by a binary neutron star (\gls{bns}) system \cite{gw170817}. As such it was one of the most interesting signals detected so far. Not only is it the lone signal one of its kind observed yet, but we were also able to detect an electromagnetic (\gls{em}) counterpart. This allowed to localize the source very precisely and get a detailed frequency evolution of the \gls{em} radiation emitted, thus helping to understand the inert dynamics of neutron stars.\\
To detect an \gls{em} counterpart astronomers need to be alerted quickly when the detector registers a possible \gls{bns} signal. To put the time scales involved into perspective, the $\gamma$-ray burst detected by Fermi-GBM arrived only \SI{1.7}{\s} after the \gls{gw} \cite{gw170817}. To reduce latency as much as possible and maximize observation time the detection pipeline needs to generate reliable triggers in as close to real time as possible. Alongside the trigger itself an estimate of the sky position needs to be provided as well.\\
The current pipeline uses the concept of matched filtering where a fixed number of pre-calculated \gls{gw} templates are used to search for similar patterns in the detector data. \textcolor{red}{[Citation]} These templates cover some area of the parameter space of binary systems we expect to detect. The sensitivity, however, is directly connected to the spacing of templates in this high dimensional parameter space. As our knowledge about binary systems and their dynamics improves and as we develop ever more accurate waveform models, this template bank will need to be expanded. This is especially true when detectors with greater sensitivity are considered. The downside of an increased template bank is the computational cost as it directly scales with the number of waveforms. It might therefore not be feasible to use matched filtering under consideration of the full template bank.
\textcolor{red}{Just mention Huerta and their pioneering work. Talk about how they only did it for BBH signals and we are trying to expand their concept to BNS case, which is partially more interesting for the reasons above. Mention that a final goal would be to have an algorithm that can compare to matched filtering in sensitivity but at fixed computational cost. Mention that computational cost is shifted from evaluation stage to training stage. Elude that sky location estimation would be final goal, but we can't get there yet. Describe what our network takes as input and what it outputs.}
%The third observation run is currently active and has already found multiple \gls{gw} candidates, many of which will probably become confirmed detections. With \gls{gw}s slowly becoming a standard tool in astronomy, astronomers need to be notified quickly and accurately to search for a possible electromagnetic counterpart. The current setup to generate these notifications is computationally expensive and doesn't scale well with the ever improving detectors. Therefore, a quick and reliable pipeline to flag the detector in real time would be beneficial. 