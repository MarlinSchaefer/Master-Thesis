\section{Introduction}
\textcolor{Blue}{Text for introduction. State where the source code used in this work can be found, list which software versions were used.}\\
With the first direct detection of a gravitational wave (\gls{gw}) on September the 14th 2015 \cite{gw150914} the age of gravitational wave astronomy began. It opened up the possibilities to test Einstein's theory of gravity in highly relativistic systems \cite{test_gr_gw150914}, sample the population of compact binary systems consisting of objects like neutron stars or black holes \cite{population_binary_systems}, define new atronomical standard candles \cite{standard_candles} and many more. The first and second observation run of the advanced LIGO and Virgo detectors \cite{aligo, avirgo} led to 11 detections of \gls{gw}s from different systems \cite{catalog}. The third observation run, which is currently ongoing, promises to greatly expand this catalog and has already found multiple \gls{gw} candidates.\\
The most promising source of \gls{gw}s are binary systems consisting of two very compact objects like black holes or neutron stars. So far all confirmed detections of \gls{gw}s were caused by such compact binary systems. Most of them were generated by systems of two coalescing black holes, the only exception being GW170817 \cite{catalog}. GW170817 was instead emitted by a binary neutron star (\gls{bns}) system \cite{gw170817}. As such it was one of the most interesting signals detected so far. Not only is it the lone signal one of its kind observed so far, but we were also able to detect an electromagnetic counterpart. This allowed to localize the source very precisely and get a detailed frequency evolution of the electromagnetic radiation emitted, thus helping to understand the inert dynamics of neutron stars. The possibility of detecting these counterpart thus makes \gls{bns} signals one of the most interesting \gls{gw} events.\\
\textcolor{red}{Talk about that BNS signals are harder to detect and that quick trigger generation is necessary to catch these counterparts. (Maybe quote how much time passed between GW and counterpart for GW170817?) Then transition to previous work and what we are trying to achieve. Also obviously talk about the current setup being too expensive and find quotes for that statement.}
%The third observation run is currently active and has already found multiple \gls{gw} candidates, many of which will probably become confirmed detections. With \gls{gw}s slowly becoming a standard tool in astronomy, astronomers need to be notified quickly and accurately to search for a possible electromagnetic counterpart. The current setup to generate these notifications is computationally expensive and doesn't scale well with the ever improving detectors. Therefore, a quick and reliable pipeline to flag the detector in real time would be beneficial. 