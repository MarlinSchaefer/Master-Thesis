\section{Introduction}
%\textcolor{Blue}{Text for introduction. State where the source code used in this work can be found, list which software versions were used.}\\
With the first direct detection of a gravitational wave (\gls{gw}) on September the 14th 2015 \cite{gw150914} the age of gravitational wave astronomy began. It opened up the possibilities to test Einstein's theory of gravity in highly relativistic systems \cite{test_gr_gw150914}, sample the population of compact binary systems consisting of objects like neutron stars or black holes \cite{population_binary_systems}, define new astronomical standard candles \cite{standard_candles} and many more. The first and second observation run of the advanced LIGO and Virgo detectors \cite{aligo, avirgo} led to 11 detections of \gls{gw}s from different systems \cite{catalog}. The third observation run, which is currently ongoing, promises to greatly expand this catalog and has already found multiple \gls{gw} candidates.\\
The most promising source of \gls{gw}s are binary systems consisting of two black holes, neutron stars or a mix of these two. So far all confirmed detections of \gls{gw}s were caused by such compact binary systems. Most of them were generated by two coalescing black holes, the only exception being GW170817 \cite{catalog}. The latter, instead, was emitted by a binary neutron star (\gls{bns}) system \cite{gw170817}. As such it was one of the most interesting signals detected so far. Not only is it the lone signal of its kind observed yet, but we were also able to detect an electromagnetic (\gls{em}) counterpart. This allowed to localize the source very precisely and get a detailed frequency evolution of the \gls{em} radiation emitted, thus helping to understand the inert dynamics of neutron stars.\\
To detect an \gls{em} counterpart astronomers need to be alerted quickly when the detector registers a possible \gls{bns} signal. To put the time scales involved into perspective, the $\gamma$-ray burst detected by Fermi-GBM arrived only \SI{1.7}{\s} after the \gls{gw} \cite{gw170817}. To reduce latency as much as possible and maximize observation time the detection pipeline needs to generate reliable triggers in as close to real time as possible. Alongside the trigger itself an estimate of the sky position needs to be provided as well.\\
The current pipeline uses the concept of matched filtering where a fixed number of pre-calculated \gls{gw} templates are used to search for similar patterns in the detector data. \textcolor{red}{[Citation]} These templates cover some area of the parameter space of binary systems we expect to detect. The sensitivity, however, is directly connected to the spacing of templates in this high dimensional parameter space. As our knowledge about binary systems and their dynamics improves and as we develop ever more accurate waveform models, this template bank will grow in size. This is especially true when detectors with greater sensitivity are considered. The downside of an increased size of the template bank is the computational cost associated with it. The CPU time scales directly with the number of waveforms that need to be compared with the data. Therefore, in the future it might not be feasible to use matched filtering under consideration of the full template bank as the only trigger generator. \textcolor{red}{(Are there other trigger generators in use already?)}\\
One of the possible contenders to aid matched filtering in the first data analysis stage is machine learning, a field of computer science that aims to create computer programs which adapt to a problem without direct human interference, i.e. learning from a set of experiences. Most of today's state of the art machine learning algorithms are implementations of neural networks (\gls{nns}). They found application in many fields, like computer vision \cite{ILSVRC15}, sound generation or natural language processing \cite{natural_language_processing}. The advantages of \gls{nns} are manifold, one of them being computational efficiency once the network is optimized. This and their general success in almost any area makes \gls{nns} interesting for \gls{gw} data-analysis.\\
Daniel George and E.A. Huerta were the first to apply a deep \gls{nn} to whitened time series strain data to try and recover \gls{gw} signals. They were able to reach performances comparable to those of matched filtering at a fraction of the computational cost \cite{original_deep_filtering}. Their network, however, was only optimized for signals from binary black hole \gls{bbh} mergers, thus not covering the cases where quick notifications are most valuable.\\
This thesis builds on the work of \cite{original_deep_filtering} and tries to expand it to \gls{bns} signals. Detecting \gls{gw}s from two coalescing neutron stars using a \gls{nn} is more challenging as these signals tend to be weaker, contain higher frequencies and are within the sensitive frequency-region of the detectors for longer time periods. Thus we introduce a novel approach to handle longer time series by using multiple sample rates. To get a first hold of the problem we will be ignoring spins and tidal deformabilities of the neutron stars. Our algorithm will be able to take a continuous stretch of strain amplitude time series data and generate from it two output time series; a signal-to-noise ratio (\gls{snr}) time series and p-score\footnote{The p-score must not be confused with a p-value. Both have in common that they are normalized to 1. The p-score however does not fulfill any other requirements and is thus not a probability.} time-series. To both of these a threshold at fixed false-alarm rate will be applied to generate triggers.\medskip\\
This thesis is structured as follows. \Cref{sec:gravitational_waves} and \autoref{sec:neural_networks} will give a brief summary of the required background knowledge. Basic knowledge of general relativity and linear algebra are assumed. \Cref{sec:related_works} will give a deeper motivation to the problem we are trying to analyze and puts this thesis into greater context of related works. \Cref{sec:network_topologies} contains our research and will go into detail about the design decisions that went into our final network. It will furthermore outline the design process and evaluate the resulting architecture on long stretches of data to compare this novel approach with existing methods.\medskip\\
To design and train our networks we use version 2.2.4 of the software library Keras \cite{keras}. The former is a wrapper for the deep learning library Tensorflow \cite{tensorflow}, of which we use the GPU optimized version 1.13.1 for training. To evaluate our networks we use version 1.14.0 of the CPU based implementation of Tensorflow. To generate fake data we use version 1.13.5 of the software package PyCBC \cite{pycbc}. \textcolor{red}{(Remove later quotation of the same)}. All code related to this thesis is open source and can be found at \url{https://github.com/MarlinSchaefer/master_project}.
%\textcolor{red}{Just mention Huerta and their pioneering work. Talk about how they only did it for BBH signals and we are trying to expand their concept to BNS case, which is partially more interesting for the reasons above. Mention that a final goal would be to have an algorithm that can compare to matched filtering in sensitivity but at fixed computational cost. Mention that computational cost is shifted from evaluation stage to training stage. Elude that sky location estimation would be final goal, but we can't get there yet. Describe what our network takes as input and what it outputs.}
%The third observation run is currently active and has already found multiple \gls{gw} candidates, many of which will probably become confirmed detections. With \gls{gw}s slowly becoming a standard tool in astronomy, astronomers need to be notified quickly and accurately to search for a possible electromagnetic counterpart. The current setup to generate these notifications is computationally expensive and doesn't scale well with the ever improving detectors. Therefore, a quick and reliable pipeline to flag the detector in real time would be beneficial. 