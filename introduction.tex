\section{Introduction}
%\textcolor{Blue}{Text for introduction. State where the source code used in this work can be found, list which software versions were used.}\\
With the first direct detection of a gravitational wave (\gls{gw}) on September the 14th 2015 \cite{gw150914}, the age of gravitational wave astronomy began. It opened up the possibilities to test Einstein's theory of gravity in highly relativistic systems \cite{test_gr_gw150914}, sample the population of compact binary systems consisting of objects like neutron stars or black holes \cite{population_binary_systems}, define new astronomical standard candles \cite{standard_candles} and many more for the first time. The first and second observation runs of the advanced LIGO and Virgo detectors \cite{aligo, avirgo} led to 11 detections of \gls{gw}s from different systems \cite{catalog}. The third observation run, which is currently ongoing, promises to greatly expand this catalog and has already found multiple \gls{gw} candidates \cite{o3_alerts}.\\
The most promising source of \gls{gw}s that can be detected are binary systems consisting of two black holes, neutron stars or a mix of these two. So far, all confirmed detections of \gls{gw}s were caused by such compact binary systems. Most of them were generated by two coalescing black holes. The only exception is GW170817 \cite{catalog}, which, instead was emitted by a binary neutron star (\gls{bns}) system \cite{gw170817}. As such, it is one of the most interesting signals detected so far. It is not only the first and only signal of its kind observed yet, but it was also possible to detect the electromagnetic (\gls{em}) counterpart. This allowed to localize the source very precisely and get a detailed frequency evolution of the \gls{em} radiation emitted, thus helping to understand the internal dynamics and structure of neutron stars \cite{multi_messanger}. As detectors become more sensitive with each technological improvement, \gls{bns} signals are expected to be detected more frequently in the future.\\
In order to detect the associated \gls{em} counterparts, astronomers need to be alerted quickly when the detector registers a possible \gls{bns} signal. To put the time scales involved into perspective, the $\gamma$-ray burst detected by Fermi-GBM arrived only \SI{1.7}{\s} after the \gls{gw} \cite{gw170817}. The source was then visible for another \SI{48}{\hour} \cite{multi_messanger}. While it is unlikely that a fast \gls{gw} detection will enable astronomers to catch the associated $\gamma$-ray burst, it will help to maximize observation time in the optical- and x-ray regime. It is therefore vital to reduce latency as much as possible and find detection candidates reliably, as telescope time is expensive. To reduce costs even further an estimate of the sky position needs to be provided as well. Some \gls{em} counterparts would also be missed if telescopes are not provided with a rough location as their field of view is often very narrow.\\
Most of the current pipelines that try to identify \gls{gw}s within the detector data use the concept of matched filtering where a fixed number of pre-calculated \gls{gw} templates are used to search for similar patterns in the detector data \cite{ligo_pipelines}. These templates cover the area of the parameter space of binary systems that are expected to exist and be detectable. The sensitivity of this search to \gls{gw}s, however, is directly dependent on the spacing of templates in this high dimensional parameter space. This is due to the fact that if a signal lies between two templates it gets assigned a lower detection statistic. As the knowledge about binary systems and their dynamics improves and as more accurate waveform models are developed, the template bank will grow in size. The downside of an increased size of the template bank is the computational cost associated with it: The CPU time scales directly with the number of waveforms that need to be compared with the data.\\% Therefore, it might not be feasible to use matched filtering under consideration of the full template bank as the main trigger generator in the future. \textcolor{red}{(Mention that current matched filtering based implementations already introduce a latency of about 16s.)}\\
One of the possible contenders to aid matched filtering in the first data analysis stage is machine learning. It is a field of computer science with the goal to create computer programs which adapt to a problem without direct human interference, i.e. learning from a set of experiences. Most of today's state of the art machine learning algorithms are implementations of neural networks (\gls{nn}s). They have application in many fields, like computer vision \cite{ILSVRC15}, sound generation \cite{wavenet} or natural language processing \cite{natural_language_processing}. The advantages of \gls{nn}s are manifold, the most important one in the context of this thesis being computational efficiency once the network is optimized. This efficiency and their general success in almost any area make \gls{nn}s a promising tool for \gls{gw} data-analysis.\\
Daniel George and E.A. Huerta were the first to apply a deep \gls{nn} to whitened time series strain data to try to recover \gls{gw} signals. They were able to reach performances comparable to those of matched filtering at a fraction of the computational cost \cite{original_deep_filtering}. Their network, however, was only optimized for signals from binary black hole (\gls{bbh}) mergers, thus not covering the cases of \gls{bns} signals where quick notifications are most valuable.\\
This thesis builds on the work of \cite{original_deep_filtering} and tries to expand it to \gls{bns} signals. Detecting \gls{gw}s from two coalescing neutron stars using a \gls{nn} is more challenging as these signals tend to have a lower strain amplitude, contain higher frequencies and are within the sensitive frequency-region of the detectors for longer time periods. Thus, we introduce a novel approach to handle longer time series by using multiple rates at which the data is sampled. To get a first hold of the problem, we are ignoring spins and tidal deformabilities of the neutron stars. We expect the effects of this approximation to be small, as spins are thought to be close to zero and the impact of deformation due to tidal effects minimal. The algorithm takes a continuous stretch of whitened strain amplitude time series data and generates two output time series from it; a signal-to-noise ratio (\gls{snr}) time series and a p-score\footnote{The p-score must not be confused with a p-value. Both have in common that they are normalized to 1 and as such the p-score gives values in the range $\left[0,1\right]$. It does, however, not fulfill any other requirements and is thus not a probability.} time-series. To both of these a threshold at fixed false-alarm rate is applied to generate detection candidates.\medskip\\
This thesis is structured as follows: \Cref{sec:gravitational_waves} and \autoref{sec:neural_networks} give a summary of the required background knowledge. Specifically, \autoref{sec:gravitational_waves} gives a brief overview of the theory involved with modeling and detecting \gls{gw}s, whereas \autoref{sec:neural_networks} describes how \gls{nn}s work and introduces the concepts needed to understand the final algorithm. \Cref{sec:related_works} gives a deeper motivation to the problem we are trying to analyze and puts this thesis into greater context of related works. \Cref{sec:network_topologies} contains the results of our research and goes into detail about the design decisions that went into our final network.\medskip\\
To design and train our networks, we use version 2.2.4 of the software library Keras \cite{keras}. The former is a wrapper for the deep learning library Tensorflow \cite{tensorflow}, of which we use the GPU optimized version 1.13.1 for training. To evaluate our networks, we use version 2.2.5 of Keras and version 1.14.0 of the CPU based implementation of Tensorflow. To generate fake data, we use version 1.13.5 of the software package PyCBC \cite{pycbc}. All code related to this thesis is open source and can be found at \url{https://github.com/MarlinSchaefer/master_project}.
\newpage
$\ $
\newpage
%\textcolor{red}{Just mention Huerta and their pioneering work. Talk about how they only did it for BBH signals and we are trying to expand their concept to BNS case, which is partially more interesting for the reasons above. Mention that a final goal would be to have an algorithm that can compare to matched filtering in sensitivity but at fixed computational cost. Mention that computational cost is shifted from evaluation stage to training stage. Elude that sky location estimation would be final goal, but we can't get there yet. Describe what our network takes as input and what it outputs.}
%The third observation run is currently active and has already found multiple \gls{gw} candidates, many of which will probably become confirmed detections. With \gls{gw}s slowly becoming a standard tool in astronomy, astronomers need to be notified quickly and accurately to search for a possible electromagnetic counterpart. The current setup to generate these notifications is computationally expensive and doesn't scale well with the ever improving detectors. Therefore, a quick and reliable pipeline to flag the detector in real time would be beneficial. 